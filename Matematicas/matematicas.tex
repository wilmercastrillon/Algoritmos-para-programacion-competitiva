\documentclass[12pt, a4paper]{article}

\usepackage[utf8]{inputenc} 
\usepackage[spanish]{babel}
\usepackage{titling}
\usepackage[left=2.54cm,right=2.54cm,top=2.54cm,bottom=2.54cm]{geometry}
\usepackage{enumerate}
\usepackage{amsmath}
\usepackage{graphicx}
\usepackage{caption}
\usepackage{multicol}
\usepackage{float}

\usepackage{listings}%-para agregar codigo-
\usepackage[usenames,dvipsnames]{color}
\usepackage{color}%------------------------

%---------------------importar codigo desde archivos cpp----------------------------
\lstloadlanguages{C++}
\lstnewenvironment{code}
	{\csname lst@SetFirstLabel\endcsname}
	{\csname lst@SaveFirstLabel\endcsname}
% general command to set parameter(s)
\lstset{
	language=C++, basicstyle=\small\ttfamily, keywordstyle=\slshape,
	emph=[1]{tipo,usa}, emphstyle={[1]\sffamily\bfseries},
	morekeywords={tint,forn,forsn},
	basewidth={0.47em,0.40em},
	columns=fixed, fontadjust, resetmargins, xrightmargin=5pt, xleftmargin=15pt,
	flexiblecolumns=false, tabsize=2, breaklines, breakatwhitespace=false, extendedchars=true,
	numbers=left, numberstyle=\tiny, stepnumber=1, numbersep=9pt,
	frame=l, framesep=3pt,
	basicstyle=\ttfamily,
	keywordstyle=\color{blue}\ttfamily,
	stringstyle=\color{magenta}\ttfamily,
	commentstyle=\color{RedOrange}\ttfamily,
	morecomment=[l][\color{OliveGreen}]{\#}
}

\lstdefinestyle{C++}{
	language=C++, basicstyle=\small\ttfamily, keywordstyle=\slshape,
	emph=[1]{tipo,usa,tipo2}, emphstyle={[1]\sffamily\bfseries},
	morekeywords={tint,forn,forsn},
	basewidth={0.47em,0.40em},
	columns=fixed, fontadjust, resetmargins, xrightmargin=5pt, xleftmargin=15pt,
	flexiblecolumns=false, tabsize=2, breaklines,	breakatwhitespace=false, extendedchars=true,
	numbers=left, numberstyle=\tiny, stepnumber=1, numbersep=9pt,
	frame=l, framesep=3pt,
    basicstyle=\ttfamily,
    keywordstyle=\color{blue}\ttfamily,
    stringstyle=\color{magenta}\ttfamily,
    commentstyle=\color{RedOrange}\ttfamily,
    morecomment=[l][\color{OliveGreen}]{\#}
}

\def\nbtitle#1{\begin{Large}\begin{center}\textbf{#1}\end{center}\end{Large}}
\def\nbsection#1{\section{#1}}
\def\nbsubsection#1{\subsection{#1}}
\def\nbcoment#1{\begin{small}\textbf{#1}\end{small}}
\newcommand{\comb}[2]{\left( \begin{array}{c} #1 \\ #2 \end{array}\right)}
\def\complexity#1{\texorpdfstring{$\mathcal{O}(#1)$}{O(#1)}}
 \newcommand\cppfile[2][]{
\lstinputlisting[style=C++,linerange={#1}]{#2}
}
%------------------------------------------------------------------------------

\newcommand{\subtitulo}[1]{\begin{center}\textbf{#1}\end{center}}

% Para que busque los archivos en una carpeta arriba
\graphicspath{{../}}
\newcommand*\lstinputpath[1]{\lstset{inputpath=#1}}
\lstinputpath{../}

%------------------------------------------------------------------------------

\title{\textbf{Matem�ticas}}
\author{Wilmer Emiro Castrill�n Calder�n}

\begin{document}
	\maketitle
	
	%<*Capitulo>
	
	\section{MCD y MCM}
	\label{matematicas:mcd_y_mcm}
	
	\subtitulo{MCD (M�ximo com�n divisor)}
	
	El m�ximo com�n divisor de un conjunto de dos o mas n�meros enteros es el m�ximo n�mero entero el cual divide a  
	todos los n�meros del conjunto sin dejar residuo, por ejemplo el m�ximo com�n divisor de $35$ y $15$ es $5$ porque  
	es el m�ximo entero el cual los puede dividir a ambos. Como abreviatura se utiliza MCD o en ingles GCD (greatest  
	common divisor).\\
	
	El calculo del MCD de dos n�meros se puede hacer de manera eficiente utilizando el algoritmo de Euclides el cual
	se puede realizar en $O(log(n))$ pasos, antes de explicarlo es necesario tener en cuenta las dos siguientes 
	propiedades de la divisibilidad ($x|y$ significa que $x$ divide a $y$ con residuo $0$).
	
	\begin{equation}
		\label{matematicas:eq:division1}
		x|y \rightarrow x| \alpha y \quad \forall \alpha \in Z
	\end{equation}
	
	\begin{equation}
		\label{matematicas:eq:division2}
		x|y \wedge x|y \pm z \rightarrow x|z
		%x|y \wedge x|z \rightarrow x| \alpha y + \beta z \quad \forall \alpha, \beta \in Z
	\end{equation}
	
	El algoritmo consiste en lo siguiente: dado dos n�meros $a$ y $b$ se realiza la divisi�n entre ambos, obteniendo un
	cociente $q$ y un residuo $r$, entonces $a = b*q_{1} + r_{1}$ con $r_{1} < b$, teniendo en cuenta que el MCD divide 
	a $a$ y $b$ entonces tambi�n divide $b*q_{1}$ (propiedad \ref{matematicas:eq:division1}), y tambi�n divide a 
	$r_1$ (propiedad \ref{matematicas:eq:division2}), el proceso se reduce a encontrar el MCD entre $b$ y $r_{1}$ 
	entonces se repite el proceso, $r_{1} = b*q_{2} + r_{2}$ con $r_{2} < r_{1}$, y as� sucesivamente hasta llegar a 
	$r_{n} = 0$, lo cual indica que $r_{n-1}$ divide a $r_{n-1}$, $r_{n-2}$, ..., $b$ y $a$, entonces $r_{n-1}$ es el 
	MCD.
	
	\begin{align*}
		a &= b*q_{1} + r_{1} & r_{1} < b\\
		b &= r_{1}*q_{2} + r_{2} & r_{2} < r_{1} \\
		r_{1} &= r_{2}*q_{3} + r_{3} & r_{3} < r_{2} \\
		...\\
		r_{n-1} &= r_{n}*q_{n+1} + 0 & 0 < r_{n} \\
	\end{align*}
	
	Por ejemplo, con $a = 35$ y $b = 15$, $a = b*2 + 5$, $r_{1} = 5$, en el siguiente paso $b = 5*3 + 0$, $r_{2} = 0$, 
	como se ha llegado a un residuo $0$, el algoritmo finaliza y la respuesta es el ultimo residuo diferente de $0$,
	entonces $MCD(35,15) = 5$.\\
	
	\subtitulo{MCM (M�nimo com�n m�ltiplo)}
	
	El m�nimo com�n m�ltiplo de un conjunto de n�meros enteros es el numero entero mas peque�o el cual es m�ltiplo de
	todos los n�meros del conjunto, por ejemplo el m�nimo com�n m�ltiplo de $35$ y $15$ es $105$, pues es el menor entero
	tal que $35|105$ y $15|105$. Como abreviatura se usa MCM o en ingles LCM (lowest common multiple).\\
	
	Para calcular el MCM tambi�n se puede utilizar el algoritmo de Euclides, pues existe una relaci�n entre el MCD
	y el MCM. entonces $MCM(a*b) = (a*b)/MCD(a,b)$, esta formula es equivalente
	a $a*(b/MCD(a,b))$ como $MCD(a,b) | b$ se puede observar que el resultado sera un entero m�ltiplo de $a$, de igual
	es equivalente a $b*(a/MCD(a,b))$ y el resultado es un entero m�ltiplo de $b$, entonces $(a*b)/MCD(a,b)$
	es m�ltiplo com�n de $a$ y $b$.\\
	
	Por ejemplo, con $a = 35$ y $b = 15$ el $MCM(a,b) = 35*15/5$ (en el ejemplo anterior se calculo el MCD 
	de $a$ y $b$), entonces el resultado es $105$.
	
	\subtitulo{Implementacion}
	
	La implementanci�n del MCD consiste en simplemente aplicar los pasos descritos para el algoritmo de Euclides. La 
	implementanci�n del MCM consiste en aplicar la formula, se recomienda realizar primero la divisi�n para evitar un  
	posible overflow al trabajar con n�meros grandes.
	
	\begin{figure}[!htb]
		\minipage{0.5\textwidth}
			\centering
			\cppfile[5-8]{Matematicas/codigos/MCD_y_MCM.cpp}
			MCD
		\endminipage
		\minipage{0.5\textwidth}
			\centering
			\cppfile[10-12]{Matematicas/codigos/MCD_y_MCM.cpp}
			MCM
		\endminipage
	\end{figure}
	
	%</Capitulo>
	
\end{document}


