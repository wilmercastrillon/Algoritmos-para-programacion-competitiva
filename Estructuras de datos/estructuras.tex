\documentclass[12pt, a4paper]{article}

\usepackage[utf8]{inputenc}
\usepackage[spanish]{babel}
\usepackage{titling}
\usepackage[left=2cm,right=2cm,top=2cm,bottom=2cm]{geometry}
\usepackage{enumerate}
\usepackage{amsmath}
\usepackage{graphicx}

\usepackage{listings}%-para agregar codigo-
\usepackage[usenames,dvipsnames]{color}
\usepackage{color}%------------------------

%---------------------importar codigo desde archivos cpp----------------------------
\lstloadlanguages{C++}
\lstnewenvironment{code}
	{%\lstset{	numbers=none, frame=lines, basicstyle=\small\ttfamily, }%
	 \csname lst@SetFirstLabel\endcsname}
	{\csname lst@SaveFirstLabel\endcsname}
\lstset{% general command to set parameter(s)
	language=C++, basicstyle=\small\ttfamily, keywordstyle=\slshape,
	emph=[1]{tipo,usa}, emphstyle={[1]\sffamily\bfseries},
	morekeywords={tint,forn,forsn},
	basewidth={0.47em,0.40em},
	columns=fixed, fontadjust, resetmargins, xrightmargin=5pt, xleftmargin=15pt,
	flexiblecolumns=false, tabsize=2, breaklines,	breakatwhitespace=false, extendedchars=true,
	numbers=left, numberstyle=\tiny, stepnumber=1, numbersep=9pt,
	frame=l, framesep=3pt,
    basicstyle=\ttfamily,
    keywordstyle=\color{blue}\ttfamily,
    stringstyle=\color{magenta}\ttfamily,
    commentstyle=\color{RedOrange}\ttfamily,
    morecomment=[l][\color{OliveGreen}]{\#}
}

\lstdefinestyle{C++}{
	language=C++, basicstyle=\small\ttfamily, keywordstyle=\slshape,
	emph=[1]{tipo,usa,tipo2}, emphstyle={[1]\sffamily\bfseries},
	morekeywords={tint,forn,forsn},
	basewidth={0.47em,0.40em},
	columns=fixed, fontadjust, resetmargins, xrightmargin=5pt, xleftmargin=15pt,
	flexiblecolumns=false, tabsize=2, breaklines,	breakatwhitespace=false, extendedchars=true,
	numbers=left, numberstyle=\tiny, stepnumber=1, numbersep=9pt,
	frame=l, framesep=3pt,
    basicstyle=\ttfamily,
    keywordstyle=\color{blue}\ttfamily,
    stringstyle=\color{magenta}\ttfamily,
    commentstyle=\color{RedOrange}\ttfamily,
    morecomment=[l][\color{OliveGreen}]{\#}
}

\def\nbtitle#1{\begin{Large}\begin{center}\textbf{#1}\end{center}\end{Large}}
\def\nbsection#1{\section{#1}}
\def\nbsubsection#1{\subsection{#1}}
\def\nbcoment#1{\begin{small}\textbf{#1}\end{small}}
\newcommand{\comb}[2]{\left( \begin{array}{c} #1 \\ #2 \end{array}\right)}
\def\complexity#1{\texorpdfstring{$\mathcal{O}(#1)$}{O(#1)}}
 \newcommand\cppfile[2][]{
\lstinputlisting[style=C++,linerange={#1}]{#2}
}
%%------------------------------------------------------------------------------

\newcommand{\subtitulo}[1]{\begin{center}\textbf{#1}\end{center}}

\title{\textbf{Estructuras de datos}}
\author{Wilmer Emiro Castrillón Calderón}

\begin{document}
	\maketitle
	
	\section{Tablas aditivas}
	Son estructuras de datos utilizadas para realizar operaciones acumuladas sobre un conjunto de datos estáticos
	en un rango específico, es decir, ejecutar una misma operación(como por ejemplo la suma) sobre un intervalo
	de datos, se asume que los datos en la estructura no van a cambiar. Estas estructuras también son conocidas como
	\textit{Cumulative Frequency Table} o de manera mas informal \textit{Tablas aditivas}, aunque no necesariamente 
	son exclusivas para operaciones de suma, pues la idea general es aplicable a otras operaciones.
	
	Durante el cálculo de una misma operación sobre diferentes rangos se presenta superposición de problemas, las 
	tablas aditivas son utilizadas para reducir la complejidad computacional aprovechando estas superposiciones
	utilizando tablas de memorización.	
	
	\subtitulo{Ejemplo inicial.}
	
	Dado un vector V = \{5,2,8,2,4,3,1\} encontrar para múltiples consultas la suma de todos los elementos en un rango
	[i,j], indexando desde 1, por ejemplo con el rango [1, 3] la suma es [5+2+8] = 15.
	
	La solución trivial es hacer un ciclo recorriendo el vector entre el intervalo [i,j], en el peor de los casos se
	debe recorrer todo el vector, esto tiene una complejidad $O(n)$ puede que para una consulta sea aceptable, pero
	en casos grandes como por ejemplo un vector de tamaño $10^{5}$ y una cantidad igualmente de consultas el tiempo 
	de ejecución se hace muy alto, por lo tanto se hace necesario encontrar una mejor solución.	
	
	Como un aspecto clave se puede observar que la solución para la consulta [1,4] y la consulta [2,5] ambas tienen 
	un intervalo en común, es decir, existe una superposición de problemas en el rango [2,4], otro aspecto clave es 
	que el problema se puede reescribir de la siguiente manera: sea $suma(x)$ = $\sum_{k=1}^{x} V_{k}$ entonces:
	$\sum_{k=i}^{j} V_{k} = suma(j) - suma(i-1)$ cuando $i \neq 1$. Ahora pre-calculando $suma(x)$ se puede dar 
	una solución inmediata a cada consulta, esto se puede resolver utilizando un enfoque básico de programación 
	dinámica
	
	Para encontrar $suma(x)$ se puede reescribir como: $suma(x) = V_{x} + suma(x-1)$ con caso base 
	$suma(1)=V_{1}$, de esta manera se tendría la siguiente solución en C++
	\cppfile[6-14]{codigos/tablas_aditivas.cpp}

	\subtitulo{Suma de subconjuntos:}
	Ejemplo de subtitulo
	
	\section{Bibliografia}
	
	http://trainingcamp.org.ar/anteriores/2017/clases.shtml.\\
	http://programacioncompetitivaufps.github.io/\\
	https://www.geeksforgeeks.org/dynamic-programming-subset-sum-problem/\\
	libro: competitive programming 3.\\



\end{document}



